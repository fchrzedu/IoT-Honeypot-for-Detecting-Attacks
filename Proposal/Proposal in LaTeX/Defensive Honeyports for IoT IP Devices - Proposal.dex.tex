% Many aspects of this template can be adjusted to
% suit your own preferences. Some parts must remain as they are
% and some parts you should change. Those parts that must remain
% as they are currently are marked with a comment in the LaTeX,
% as are the parts that you should change.

% Keep the font and paper size, but you can change the class and font encoding
% Note: not all classes use \chapter.



\documentclass[a4paper,12pt,oneside]{book}			% sets doc type, book good for chapters
\usepackage[utf8]{inputenc}						% use UTF-8
\usepackage[UKenglish]{babel}						% Ensures proper formatting for English
\usepackage{lmodern}							% Loads latin Modern
\usepackage[LGR,T1]{fontenc}					% LGR = greek letters, T1 = accented Latin letters

% Some common packages, and removal of these will likely break compilation
\usepackage{amsthm,amsmath,amssymb,marvosym} 	% Only really needed if you use maths
\usepackage{graphicx} 							% for including graphics
\usepackage[dvipsnames,svgnames,x11names]{xcolor}	
\usepackage{pdflscape} % for landscape pages
\usepackage{pdfpages} % to insert pages from PDFs


\usepackage[top=2.5cm,						% Define magin spacing
            bottom=2.5cm,
            left=3.2cm,
            right=3.2cm,
            includefoot,
            footskip=30pt,
            ]{geometry}


% \usepackage{lipsum}							used for dummy text i.e. \lipsum[1-3]

% Setting the page layout (headers/footers/etc).
% This can be changed, but ensure that every page of the
% mainmatter is numbered.
\usepackage{fancyhdr}							% Customises headers and footers
\fancyhead[R]{\textsl{\rightmark}}					% Displays current section title in italic in top right
\fancyhead[L]{}								% Leave rop left corner blank
\cfoot{\thepage}								% Place page number in centre of footer
\setlength{\headheight}{15pt}						% Vertical spacing for header

% Adding clickable references and citations
% Also adding PDF metadata and so here you will want
% to change the title and author!
\usepackage[
    pdftex,
    pdftitle={Defensive Honeypot for IoT Devices - Proposal}, 
    pdfauthor={Franek Kruczynski}, 
    pdfkeywords={},
    pdfproducer={LaTeX with hyperref},
    pdfcreator={PDFLaTeX},
    pdfencoding=auto,
    psdextra,
    ]{hyperref}

% Adding the references section (using BibLaTeX).
% Feel free to adjust the style:
%   - ieee (for this style use \cite{name_of_ref})
%   - authoryear (for this style use a combination of \cite{name_of_ref} and \citep{name_of_ref})
\usepackage[
    backend=biber,
    natbib=true,
    style=authoryear,
    ]{biblatex}
\renewcommand\nameyeardelim{, }
\addbibresource{mybib.bib}
\usepackage{csquotes}

% By changing "colorlinks" to true, the boxed link text will
% change the appropriate colour instead of being surrounded
% by a box.
% The colours can be adjusted as you please and there has not
% been much thought put into this colour scheme!
\hypersetup{
    colorlinks=true,
    linkcolor=blue!50!cyan,
    filecolor=magenta,
    urlcolor=blue,
    citecolor=red,
    bookmarksopen=true
    }

% One and half line spacing (this must remain the same)
\linespread{1.25}

% Details to make the title page.
% Change these details!
\title{\huge\bfseries Defensive Honeypots for IP IoT Devices:\\Quantitative  Comparison between Vanilla and Sandboxed Honeypots}
\author{\LARGE Franek Kruczynski}
\date{September 2025}


\begin{document}

\frontmatter			% Uses roman numeral numbering in page number footer
\maketitle				% Creates title from preamble
\setcounter{page}{1}		% reset page counter for contents table
\pagestyle{fancy}

\tableofcontents 			% Creates tableofcontents

\mainmatter 			% Changes page numbering to numbers
\clearpage			




% actual text -----------------------------------
\chapter{Introduction}\label{ch:intro}		% Introduction heading
\section{Background}\label{sec:background}	% Background subheading

Abstract of the project goes here

The Internet of Things (IoT) is vastly expanding, driving a brand new and complex wave of device inter-connectivity worldwide, with an approximate 27-billion devices by the end of 2025\textit{\citep{autobits2025iot}}.

\section{Aims \&{} Objectives}\label{sec:aimAndObjectives}

\subsection{Aim}\label{sec:aim}

To evaluate the effectiveness of containment and sandboxing mechanisms, in preventing malware propagation (specifically its spread into external systems) within a IoT IP device Honeypot framework. Such will be achieved through quantitatively comparing the same malware programs on two separate Honeypots, contained high-interaction against vanilla low-interaction. 


\subsection{Objectives}\label{sec:objectives}
The objectives are as follows:
\begin{itemize}

	\item Design, develop and deploy a secure Honeynet framework within Virtual Machines,
	\item To deploy two separate Honeypots:
		\begin{description}
		\item[1.] \textbf{Vanilla Honeypot:} Low-interaction with no advanced security,
		\item[2.] \textbf{Sandboxed Honeypot:} High-interaction within a secured container.
		\end{description}
	\item To create and design a virtual network, providing both logical addressing to all IoT IP devices, Virtual Machines and, providing security through sub-netting. In essence mimicking a small office network.
	\item To collect and store the following malware properties for quantitative comparison and analysis:
		\begin{description}
		\item[1.] Network traffic,
		\item[2.] Payloads,
		\item[3.] Malware type
		\item[4.] Activity data
		\item[5.] Propagation attempts outside the container.
		\end{description}
	\item Quantitatively compare the data of all malware, and conclude whether attack behaviors differ based on environment. 
\end{itemize}

\section{Product Review}\label{sec:productReview}

\subsection{Scope}\label{sec:scope}

The project involves the design and deployment of a secure IoT Honeynet environment for a range of IP IoT devices simulated within a small office network. Two separate and distinct honeypots will be implemented, a high-interaction sandboxed Honeypot and, a low-interaction vanilla Honeypot operating within a secure, isolated Docker container. The Honeypots will operate within a singular Virtual Machine sequentially.

Each malware sample will be executed within the Honeypots, with all relevant data including its type, payloads and various activity logs securely collected. A dedicated Virtual Machine (separate to the Honeynet) will store and process the data using tools such as ElasticSearch, guaranteeing all data remains isolated from the external network.

In essence, the project will simulate realistic attacking behaviours and sequences of events that occur during malware spread within smaller networks. Findings will lead to a better understanding of the importance of segmentation within networks and Honeypots, to aid in strengthening IoT security by identifying attacker patterns \textit{\citep{Kocaogullar2023honeypots}}. Furthermore, it’ll help expand on pre-existing research and highlight the importance of practical implementation methods for secure environments.  

\subsection{Audience}\label{sec:audience}

The primary audience for this project involves cyber security researchers, network analysists and academic institutions which study digital forensics, analyse malware behaviour and the administrators that deal with IoT security. Such groups will benefit from the projects structured and quantitative evaluation of containment-based Honeypots and networks, as it provides improved threat detection and response; enhanced threat intelligence through dynamic analysis and prevention; and the overall impact of a secure Honeypot environment. 

Furthermore, Cybersecurity lecturers and students will benefit through providing a practical framework for a safe and secure virtual environment for malware deployment, for experimentation and research. The projects architecture may server as a teaching tool for demonstrating both the dangers of types of malware, and analysis practices for security.

Lastly, IoT administrators and developers in smaller organisations may gauge a better understanding of malware behaviour within contained and non-contained environments. Understanding malware patterns with supporting data may help in more advanced intrusion detection systems (IDS), through ensuring comprehensive flaws within IoT devices are identified \textit{\citep{fortinet-IDS}}. 


\chapter{Background Review}\label{ch:backgroundReview}

\section{Existing Approaches}\label{sec:existingApproaches}

Add on to 1.1, provide overview of similar products and why they aren't sufficient

\section{Related Literature}\label{sec:relatedLiterature}

Self explanatory

- Look through thesis provided by supervisor


\chapter{Methodology \&{} Techniques}\label{ch:methods}
\section{Approach}\label{sec:approach}
- Link back to objectives?

- Two separate VMs

- Lab VM = honeypots

Analysis VM = protected

\section{Technologies}\label{sec:technologies}
\section{Version Control \&{} Management}\label{sec:versionControl}
Introduce GitHub \&{} Supervisor Google Drive


\chapter{Project Management}\label{ch:project management}
\section{Activities}\label{sec:activities}
\section{Schedule and Time Management}\label{sec:time management}
- Calendar
- Allocating times during week
\section{Data Management}\label{sec:data management}
- How is this data going to be stored? (Analysis VM using pcaps)
- CSV files for extracting
\section{Deliverables}\label{sec: deliverables}



\chapter{References}

\printbibliography
\end{document}









