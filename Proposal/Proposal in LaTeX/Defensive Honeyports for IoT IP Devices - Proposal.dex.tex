% Many aspects of this template can be adjusted to
% suit your own preferences. Some parts must remain as they are
% and some parts you should change. Those parts that must remain
% as they are currently are marked with a comment in the LaTeX,
% as are the parts that you should change.

% Keep the font and paper size, but you can change the class and font encoding
% Note: not all classes use \chapter.



\documentclass[a4paper,12pt,oneside]{book}			% sets doc type, book good for chapters
\usepackage[utf8]{inputenc}						% use UTF-8
\usepackage[english]{babel}						% Ensures proper formatting for English
\usepackage{lmodern}							% Loads latin Modern
\usepackage[LGR,T1]{fontenc}					% LGR = greek letters, T1 = accented Latin letters

% Some common packages, and removal of these will likely break compilation
\usepackage{amsthm,amsmath,amssymb,marvosym} 	% Only really needed if you use maths
\usepackage{graphicx} 							% for including graphics
\usepackage[dvipsnames,svgnames,x11names]{xcolor}	
\usepackage{pdflscape} % for landscape pages
\usepackage{pdfpages} % to insert pages from PDFs


\usepackage[top=2.5cm,						% Define magin spacing
            bottom=2.5cm,
            left=3.2cm,
            right=3.2cm,
            includefoot,
            footskip=30pt,
            ]{geometry}


% \usepackage{lipsum}							used for dummy text i.e. \lipsum[1-3]

% Setting the page layout (headers/footers/etc).
% This can be changed, but ensure that every page of the
% mainmatter is numbered.
\usepackage{fancyhdr}							% Customises headers and footers
\fancyhead[R]{\textsl{\rightmark}}					% Displays current section title in italic in top right
\fancyhead[L]{}								% Leave rop left corner blank
\cfoot{\thepage}								% Place page number in centre of footer
\setlength{\headheight}{15pt}						% Vertical spacing for header

% Adding clickable references and citations
% Also adding PDF metadata and so here you will want
% to change the title and author!
\usepackage[
    pdftex,
    pdftitle={Defensive Honeypot for IoT Devices - Proposal}, 
    pdfauthor={Franek Kruczynski}, 
    pdfkeywords={},
    pdfproducer={LaTeX with hyperref},
    pdfcreator={PDFLaTeX},
    pdfencoding=auto,
    psdextra,
    ]{hyperref}

% Adding the references section (using BibLaTeX).
% Feel free to adjust the style:
%   - ieee (for this style use \cite{name_of_ref})
%   - authoryear (for this style use a combination of \cite{name_of_ref} and \citep{name_of_ref})
\usepackage[
    backend=biber,
    natbib=true,
    style=authoryear,
    ]{biblatex}
\renewcommand\nameyeardelim{, }
\addbibresource{mybib.bib}
\usepackage{csquotes}

% By changing "colorlinks" to true, the boxed link text will
% change the appropriate colour instead of being surrounded
% by a box.
% The colours can be adjusted as you please and there has not
% been much thought put into this colour scheme!
\hypersetup{
    colorlinks=false,
    linkcolor=blue!50!cyan,
    filecolor=magenta,
    urlcolor=blue,
    citecolor=red,
    bookmarksopen=true
    }

% One and half line spacing (this must remain the same)
\linespread{1.25}

% Details to make the title page.
% Change these details!
\title{\huge\bfseries This is the title for the proposal}
\author{Franek Kruczynski}
\date{September 2025}


\begin{document}

\frontmatter			% Uses roman numeral numbering in page number footer
\maketitle				% Creates title from preamble
\setcounter{page}{1}		% reset page counter for contents table
\pagestyle{fancy}

\tableofcontents 			% Creates tableofcontents

\mainmatter 			% Changes page numbering to numbers
\clearpage			




% actual text -----------------------------------
\chapter{Introduction}\label{ch:intro}		% Introduction heading
\section{Background}\label{sec:background}	% Background subheading

Abstract of the project goes here

\section{Aims \&{} Objectives}\label{sec:aimAndObjectives}
\subsection{Aim}\label{sec:aim}

Research Question and aim of dissertation 

\subsection{Objectives}\label{sec:objectives}

Objectives required to meet 1.2.1

\section{Product Review}\label{sec:productReview}

\subsection{Scope}\label{sec:scope}

What will the project do?

What is its purpose ?

How will it work?


\subsection{Audience}\label{sec:audience}

Who is this project for?

\chapter{Background Review}\label{ch:backgroundReview}

\section{Existing Approaches}\label{sec:existingApproaches}

Add on to 1.1, provide overview of similar products and why they aren't sufficient

\section{Related Literature}\label{sec:relatedLiterature}

Self explanatory


\chapter{Methodology \&{} Techniques}\label{ch:methods}
\section{Approach}\label{sec:approach}
\section{Technologies}\label{sec:technologies}
\section{Version Control \&{} Management}\label{sec:versionControl}
Introduce GitHub \&{} Supervisor Google Drive


\chapter{Project Management}\label{ch:project management}
\section{Activities}\label{sec:activities}
\section{Schedule and Time Management}\label{sec:time management}
\section{Data Management}\label{sec:data management}
\section{Deliverables}\label{sec: deliverables}
\end{document}









